\section{Abstract}

\begin{description}
\item[Principal Investigator] Carl Brandon
\item[University] Vermont Technical College
\end{description}

The research objectives of mission are as follows: the first objective
is to adapt the CubedOS system to swarms of cubesats, which have low
computational and communication capabilities, the second is to create
fuel-optimal trajectory planning and collision-free real-time control
for satellite swarms in LEO through the development of algorithms by
the University of Vermont. It is important to note that there is
synergy between objective two, and a project that NASA is already
funding\cite{ossareh:2019}.

Over the next two years, Both Vermont Technical College (VTC) and the
University of Vermont (UVM) will be jointly developing these
overarching goals. This mission will seek to modify the Cubed OS
system, running on SPARK Ada to accommodate the swarm behavior listed
above. This will culminate in a demonstration of a CubeSat swarm
totaling three 1U CubeSats in LEO using a combination of magnetic
torquers and hysteresis rods for attitude control via adjustment and
breaking respectively. The control system algorithms for the swarm
will be developed by the University of Vermont in Simulink, and
integrated into CubedOS and evaluated for the demonstration.

This inter-university collaboration effort is to be broken down as
follows: UVM will be developing the algorithms for the swarm including
the collision-free real-time control systems, VTC will be using their
software platform to develop and test the program and the assembly of
the hardware will be developed jointly. This is also currently
combined with the interest in researching non-toxic, small scale
thruster systems for maneuvering systems via Benchmark Space Systems.

The intention is to launch the swarm into orbit, and use the VTC
ground station, used for the previous 2012 mission, Vermont Lunar, to
communicate with the swarm, in order to exchange information and
update configuration orders if need be.
	
The anticipated outcome is to build a foundation for any future
missions using CubedOS for single-sat or swarm-sat configurations and
to further integrate autonomous systems in the space industry through
this adaptation, as well to increase inter-university cooperation and
student interest both in CubeSat hardware and software development,
while also paving the way for future university missions.
	
The impacts of this research are wide reaching, allowing for greater
autonomy in fleets of potential probes, and other deep space machines,
which require the ability to make complex decisions, such as collision
avoidance, without human intervention, as communication time increases
as a function of irreducible distance. This will also provide a
modular software platform for other CubeSat missions to build their
own software from, whether university, DoD, NASA, or privately based.
