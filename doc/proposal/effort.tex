\section{Research Effort}

In a CubeSat flying formation, one of the most significant challenges
is maintaining proper attitude to ensure correct orbital trajectories
and thus proper functioning of the swarm. Traditionally, attitude
control is carried out by reaction wheels or magnetorquer rods.

Reaction wheels can be accurate but volume intensive and often fail
over time due to the constant rotation and high torques. Magnetorquer
rods do not require moving parts but provide less torque and still
take up substantial space. One of the key aspects of this project is
to design, test and fly solar panel PCB-integrated magnetorquers which
will be manufactured by partner company LED Dynamics who provided
solar panels for the VT Lunar CubeSat mission.

By integrating magnetorquers directly into the solar panels, their
volume will be effectively zero while still providing the necessary
torque for attitude control. The research will be based upon existing
literature which describes in detail the process of optimizing
geometry (i.e. trace width, number of traces) given initial electric
inputs (e.g. constant current, constant voltage)(\cite{ali:2021},
\cite{khan:2022}, \cite{sokal:2019}, \cite{sorensen:2021}).

 The general process will be to design a magnetorquer geometry based
 on the mission requirements and prior art, then submit that design to
 LED Dynamics for fabrication. The research will continue with the
 launch of the CubeSats where the efficacy of the magnetorquers in
 maintaining attitude throughout the formation change will be analyzed
 and corrected by the autonomous control algorithms. The control
 system will be integrated into a board that will house the IMU and
 Lithium radio.

This path will open up the use of combination torquer-hysteresis for
attitude adjustment and control for future CubeSat missions, as a
means of direction control in LEO.

The CubeSats will also be utilizing the JT65 radio communication
protocol, which has been modified to be controlled autonomously by
CubedOS instead of by human operators. The JT65 protocol has several
advantages that make it a promising technology for deep-space CubeSat
missions. It is a low data rate protocol that produces error free
communications and allows for CubeSats to be controlled from as far as
Jupiter \cite{brandon:2019}. The project aims to provide continued
proof that the JT65 is a suitable communication method, building upon
its success in the VT Lunar CubeSat mission. By avoiding the steep
cost and limited schedule of NASA’s Deep Space Network (DSN) CubeSat
missions far from LEO could achieve lower costs with the same
information transfer via this method.

Looking towards the future of the technology to be implimented in this
mission its concievable influence may be given a broad brush via large
scale orbital construction, and resource mining.

This research benefits the future field of large-scale construction in
orbit by providing a potential software scaffold for maneuvering
multiple swarm drones. It is conceivable that these future drones can
be made to work in a 3-dimensional or multi-layered 2-dimensional
region of space, using radio, or light signaling from a stationary
drone, or set of drones. This kind of work has been done on earth,
using 2-D orbital construction protocols \cite{Vardy:2018}.

Apart from construction in orbit and beyond is the acquisition of
resources in space, namely asteroids. Given that asteroids are good
sources of volatiles, propellants, construction materials, and
precious metals \cite{Ross:2020}, there is a demand to make the
acquisition process affordable, and repeatable.

Some of the current literature insists that the Mass Payback Ratio,
startup costs, and risk associated with the current state of the art
ensures the failure of any venture. However, the article cited also
notes that the following: "We find that from a profitability
perspective, the throughput rate and using smaller but multiple
spacecraft per mission are key technical parameters for reaching
breakeven quickly" \cite{Andreas:2001}.

Thus, the introduction of a few 1U CubeSat drones, with swarm behavior
could conceivably lower the cost and risk of capture and
transportation of Near Earth Objects (NEOs), by providing relatively
cheap artificial workers to retrieve and transport objects of interest
to potential lagrange points, or resource centers in orbit for
refinement.

Both large scale construction, and NEO capture could concievably make
up the backbone of a linear refinement and production industry, where
space production is preferable to terra ferma production.


