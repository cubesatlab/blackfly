\chapter{External Interface Requirements}
\label{chap:external-interface}

\section{Command and Telemetry Dictionaries}

The user interface of a spacecraft can be described as the set of
commands that can be sent to the spacecraft and the telemetry that is
returned by the spacecraft. These are described in the form of
dictionaries.

\begin{description}
\item[{\varcmd}.Length] \question{Is there a limit to how much data
  can be in a command?}
\item[{\varcmd}.Formatting] \question{Are there any requirements about
  how commands must be packaged and formatted?}
\item[{\vartelem}.Length] \question{Is there a limit to how much data
  can be in a telemetry data unit?}
\item[{\vartelem}.Formatting] \question{Are there any requirements
  about how telemetry data units must be packaged?}
\end{description}

The command dictionary lists all the commands to which the spacecraft
must respond, along with the type and meaning of any command arguments
and the semantics of the commands. \question{Is there a standard
  format we should use to facilitate interaction with specific ground
  software or to facilitate generation by specific tools?}

TODO: Define command dictionary here (or maybe in an external document).

The telemetry dictionary lists all the telemetry data units the
spacecraft can produce, along with the type and meaning of the data
arguments. \question{Is there a standard format we should use to
  facilitate interaction with specific ground software or to
  facilitate acceptance by specific tools?}

TODO: Define telemetry dictionary here (or maybe in an external
document).

\section{Hardware Interfaces}
% Describe the logical and physical characteristics of each interface
% between the software product and the hardware components of the
% system. This may include the supported device types, the nature of
% the data and control interactions between the software and the
% hardware, and communication protocols to be used.

\section{Software Interfaces}
% Describe the connections between this product and other specific
% software components (name and version), including databases,
% operating systems, tools, libraries, and integrated commercial
% components. Identify the data items or messages coming into the
% system and going out and describe the purpose of each. Describe the
% services needed and the nature of communications.  Refer to
% documents that describe detailed application programming interface
% protocols.  Identify data that will be shared across software
% components. If the data sharing mechanism must be implemented in a
% specific way (for example, use of a global data area in a
% multitasking operating system), specify this as an implementation
% constraint.

\section{Communications Interfaces}
% Describe the requirements associated with any communications
% functions required by this product, including e-mail, web browser,
% network server communications protocols, electronic forms, and so
% on. Define any pertinent message formatting. Identify any
% communication standards that will be used, such as FTP or
% HTTP. Specify any communication security or encryption issues, data
% transfer rates, and synchronization mechanisms.
