\chapter{Overall Description}
\section{Product Perspective}
% Describe the context and origin of the product being specified in
% this SRS. For example, state whether this product is a follow-on
% member of a product family, a replacement for certain existing
% systems, or a new, self-contained product. If the SRS defines a
% component of a larger system, relate the requirements of the larger
% system to the functionality of this software and identify interfaces
% between the two. A simple diagram that shows the major components of
% the overall system, subsystem interconnections, and external
% interfaces can be helpful.

\section{Product Functions}
% Summarize the major functions the product must perform or must let
% the user perform. Details will be provided in Section 3, so only a
% high level summary (such as a bullet list) is needed here. Organize
% the functions to make them understandable to any reader of the
% SRS. A picture of the major groups of related requirements and how
% they relate, such as a top level data flow diagram or object class
% diagram, is often effective.

\section{User Classes and Characteristics}

% Identify the various user classes that you anticipate will use this
% product. User classes may be differentiated based on frequency of
% use, subset of product functions used, technical expertise, security
% or privilege levels, educational level, or experience. Describe the
% pertinent characteristics of each user class. Certain requirements
% may pertain only to certain user classes. Distinguish the most
% important user classes for this product from those who are less
% important to satisfy.

Roughly, we identify the following user classes.
\begin{enumerate}
\item The users interested in the results (telemetry) of the formation flying demonstration
  (important).
\item The CubedOS Laboratory has an interest in the performance of CubedOS (software telemetry)
  (important).
\item The amateur radio community may be interested in helping us gather telemetry and thus
  constitutes a user class in the sense that we will have requirements related to data formats,
  etc. (important).
\item Benchmark Space Systems (perhaps having interest in telemetry related to the thrusters).
\item Future users of distributed drone systems have an interest in the technology we are
  demonstrating here.
\end{enumerate}
\question{What, if anything, do the UVM students get from this project? Are they a user class?}

\section{Operating Environment}
In this section we describe the environment in which the software will
operate. Here ``environment'' refers to both the hardware and software
environment. These requirements are constraints on the software since
the software must work around them or within them.

\begin{description}
\item[{\varenv}.Processor] The software shall run on an ARM
  architecture processor.  What are the CPU details?
\item[{\varenv}.Memory] What memory constraints exist for the system?
\item[{\varenv}.Storage] What non-volatile, read/write storage
  constraints exist for the system?
\item[{\varenv}.OS] What underlying operating system will be used?
  There has been discussion at VTC about designing for a CubedOS/CFS
  hybrid system. The idea would be to use CubedOS as the primary
  environment but bridge to CFS for some minor, but still interesting
  functionality. This would allow us to demonstrate a CubedOS/CFS
  bridge (to be constructed), which would be important for generating
  interest in CubedOS in the CFS community.
\item[{\varenv}.Libraries] Are there any particular libraries that
  must (or should) be used?
\item[{\varenv}.Power] Are there any power requirements the software
  needs to support? For example: required CPU sleep time. Requirements
  about managing the power of other spacecraft components are
  specified elsewhere since those requirements are about what the
  software must do rather than being constraints on what the software
  itself must work within.
\end{description}

\section{Design and Implementation Constraints}
% Describe any items or issues that will limit the options available
% to the developers. These might include: corporate or regulatory
% policies; hardware limitations (timing requirements, memory
% requirements); interfaces to other applications; specific
% technologies, tools, and databases to be used; parallel operations;
% language requirements; communications protocols; security
% considerations; design conventions or programming standards (for
% example, if the customer’s organization will be responsible for
% maintaining the delivered software).

\begin{enumerate}
\item The system shall use standard amateur radio communications protocols so that the amateur
  radio community can assist in tracking and data gathering. \question{Are there any security
    implications related to this?}
\item ITAR restrictions must be followed.
\end{enumerate}

\section{User Documentation}
% List the user documentation components (such as user manuals,
% on-line help, and tutorials) that will be delivered along with the
% software. Identify any known user documentation delivery formats or
% standards.

\begin{enumerate}
\item The CubedOS documentation shall be updated to reflect any developments made to the kernel
  or the core modules.
\item Any CubedOS device servers shall be documented so that other mission can potentially reuse
  them.
\end{enumerate}

\section{Assumptions and Dependencies}
% List any assumed factors (as opposed to known facts) that could
% affect the requirements stated in the SRS. These could include
% third-party or commercial components that you plan to use, issues
% around the development or operating environment, or constraints. The
% project could be affected if these assumptions are incorrect, are
% not shared, or change. Also identify any dependencies the project
% has on external factors, such as software components that you intend
% to reuse from another project, unless they are already documented
% elsewhere (for example, in the vision and scope document or the
% project plan).

\begin{enumerate}
\item It is the intention to use Simulink (and MATLAB) to model the swarm algorithm and then
  generate \SPARK\ using QGen from that model.
\item This project has a dependency on various (to enumerate later) Ada/\SPARK\ tools from
  AdaCore.
\end{enumerate}
