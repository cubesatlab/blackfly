\chapter{Nonfunctional Requirements}
\label{chap:nonfunctional}

\section{Performance Requirements}

\question{How much space is necessary to buffer telemetry on the
  spacecraft?}

\question{How much space is necessary to buffer commands on the
  spacecraft?}

\question{What temporaly accuracy is required for timestamping
  telemetry?}

\question{What temporaly accuracy is required for issuing commands?}

\question{Are there any latency requirements regarding the time
  between command issuance and command completion? This might be
  better documented in the command dictionary.}

\question{Are there any CPU power consumption requirements we need to
  know about? Sleep time?}

\section{Safety Requirements}

\question{What are our de-orbiting requirements?}

\question{What does ``safe mode'' look like? For example, if there is
  failure after deployment, we want to avoid damaging other spacecraft
  in our vicinity.}

\question{How long do we have to wait after deployment before we can
  do things like fire the thrusters or deploy the antenna?}

\question{In general, what is the deployment sequence? This might be
  more properly a functional requirement.}

\section{Security Requirements}

\question{To what extent should we be concerned about people on the
  ground a) monitoring the traffic to/from the spacecraft(s), and b)
  taking control of the spacecraft(s) by sending commands of their
  own? In other words: do we need to have confidentiality or data
  integrity in the earth to spacecraft communications link? What about
  the spacecraft to spacecraft link?}

\question{Do we want to require operators on the ground to
  authenticate to the spacecraft before they can interact with it
  (i.e., send commands or receive telemetry)?}

\section{Software Quality Attributes}

In what follows the term \newterm{codebase} refers to the code that is
ultimately loaded onto the spacecraft that flies in space. This
document does not assert any quality rules for supporting code (that
does not fly in space), although it is encouraged to follow similar
rules for such code when feasible.

\begin{description}
\item[Quality.FlowErrors] The codebase shall be free of flow errors in
  the sense meant by \SPARK. Note that without this, Quality.AORTE
  can't be verified by \SPARK, even in principle.

\item[Quality.AORTE] The codebase shall be written in \SPARK\ and
  proved to be free of runtime errors in the sense meant by
  \SPARK. Certain, limited exceptions are allowed, but they shall be
  documented using justifications in the source files.

\item[Quality.Functional] We rely on conventional unit and integration
  testing to verify functional properties of the codebase. These
  requirements \emph{do not} require that functional properties be
  proved by \SPARK. However, using \SPARK\ to prove functional
  properties is encouraged where feasible, and may be required, to
  some extent, in a future version of these requirements.

\item[Quality.Testability] Library components of the codebase and
  easily factored module sub-functionality shall be tested by unit
  tests that can be executed and verified automatically. Full module
  functionality may be tested by manual tests (due to the nature of
  CubedOS modules), although finding a way to verify their behavior
  automatically would be desirable. The manual tests do not relax the
  Quality.Coverage requirement (indeed, the manual tests will be
  needed to meet the requirement).

\item[Quality.Coverage] Test coverage shall include the execution of
  every line of code in the codebase by some test, as documented by
  the output of a suitable coverage tool (such as \command{gcov}).

\item[Quality.Style] The codebase shall be written according to an
  agreed-upon style using a style checking tool to enforce that
  style. The details of the style are not specified here, but are
  instead described in the style checking tool's configuration file.

\item[Quality.Documentation] The codebase shall be documented at two
  levels: The public interface of every package shall be documented
  using a standard API documentation extraction tool such as
  \command{gnatdoc}. The higher level functionalilty of the codebase
  shall be documented using ordinary English prose.
  
\item[Quality.CI] A continuous integration server shall be configured
  to perform the following steps on at least a daily basis:
  \begin{itemize}
  \item Verify clean, warning-free compliation of the codebase and
    test programs.
  \item Verify Quality.FlowErrors
  \item Verify Quality.AORTE
  \item Verify error-free execution of the unit and integration tests
    (to the greatest extent possible... some tests might be difficult
    to automate).
  \item Verify Quality.Coverage
  \item Verify Quality.Style
  \item Extract the API documentation (see Quality.Documentation)
  \item Verify error-free compilation of the high level documentation
    (see Quality.Documenation)
  \end{itemize}

\item[Quality.Commit] All commits to the codebase shall compile
  without error, all tests shall pass, and all style rules shall pass
  or else the commit is considered ``invalid'' and must be corrected
  immediately. It is understood, however, that compile-time warnings,
  Quality.FlowErrors, Quality.AORTE, and Quality.Coverage may be
  temporarly unsatisfied during the development process.

\item[Quality.Reusability] Non-hardware specific components of the
  codebase shall be written so they could potentially be reused in
  other missions. Hardware specific components shall be written so
  they could potentially be reused in other missions using the same
  hardware.

\end{description}

\section{Business Rules}

% List any operating principles about the product, such as which
% individuals or roles can perform which functions under specific
% circumstances. These are not functional requirements in themselves,
% but they may imply certain functional requirements to enforce the
% rules.

\section{Other Requirements}

\question{Somewhere we need to specify what software telemetry is
  required. This might be better done where the telemetry dictionary
  is specified.}

\begin{description}
\item[Other.SourceFormat] Source files of the codebase shall be
  Unicode files encoded with UTF-8. Latin and Greek letters are
  allowed in identifier names (and digits, etc., but no other
  letters). This is useful for writing code that directly translates
  mathematical equations.

\item[Other.License] \question{What license should we use for the
  codebase?}

\item[Other.DataFormat] \question{Do we have requirements for the data
  format to use for telemetry data?}

\item[Other.GroundSystem] \question{To what extent are we writing
  ground system software? If we are it might need an separate
  requirements document of its own.}
\end{description}
