\chapter{System Features}
\label{chapt-system-features}

\section{System Feature 1}
% Don't really say ``System Feature 1.'' State the feature name in
% just a few words.

\subsection{Description and Priority}
% Provide a short description of the feature and indicate whether it
% is of High, Medium, or Low priority. You could also include specific
% priority component ratings, such as benefit, penalty, cost, and risk
% (each rated on a relative scale from a low of 1 to a high of 9).

\subsection{Stimulus/Response Sequences}
% List the sequences of user actions and system responses that
% stimulate the behavior defined for this feature. These will
% correspond to the dialog elements associated with use cases.

\subsection{Functional Requirements}
% Itemize the detailed functional requirements associated with this
% feature. These are the software capabilities that must be present in
% order for the user to carry out the services provided by the
% feature, or to execute the use case. Include how the product should
% respond to anticipated error conditions or invalid
% inputs. Requirements should be concise, complete, unambiguous,
% verifiable, and necessary. Use ``TBD'' as a placeholder to indicate
% when necessary information is not yet available.

\section{System Feature 2}
% Each requirement should be uniquely identified with a sequence
% number or a meaningful tag of some kind.
